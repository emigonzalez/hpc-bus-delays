%% bare_jrnl.tex
%% V1.4b
%% 2015/08/26
%% by Michael Shell
%% see http://www.michaelshell.org/
%% for current contact information.
%%
%% This is a skeleton file demonstrating the use of IEEEtran.cls
%% (requires IEEEtran.cls version 1.8b or later) with an IEEE
%% journal paper.
%%
%% Support sites:
%% http://www.michaelshell.org/tex/ieeetran/
%% http://www.ctan.org/pkg/ieeetran
%% and
%% http://www.ieee.org/

%%*************************************************************************
%% Legal Notice:
%% This code is offered as-is without any warranty either expressed or
%% implied; without even the implied warranty of MERCHANTABILITY or
%% FITNESS FOR A PARTICULAR PURPOSE! 
%% User assumes all risk.
%% In no event shall the IEEE or any contributor to this code be liable for
%% any damages or losses, including, but not limited to, incidental,
%% consequential, or any other damages, resulting from the use or misuse
%% of any information contained here.
%%
%% All comments are the opinions of their respective authors and are not
%% necessarily endorsed by the IEEE.
%%
%% This work is distributed under the LaTeX Project Public License (LPPL)
%% ( http://www.latex-project.org/ ) version 1.3, and may be freely used,
%% distributed and modified. A copy of the LPPL, version 1.3, is included
%% in the base LaTeX documentation of all distributions of LaTeX released
%% 2003/12/01 or later.
%% Retain all contribution notices and credits.
%% ** Modified files should be clearly indicated as such, including  **
%% ** renaming them and changing author support contact information. **
%%*************************************************************************


% *** Authors should verify (and, if needed, correct) their LaTeX system  ***
% *** with the testflow diagnostic prior to trusting their LaTeX platform ***
% *** with production work. The IEEE's font choices and paper sizes can   ***
% *** trigger bugs that do not appear when using other class files.       ***                          ***
% The testflow support page is at:
% http://www.michaelshell.org/tex/testflow/



\documentclass[journal]{IEEEtran}
%
% If IEEEtran.cls has not been installed into the LaTeX system files,
% manually specify the path to it like:
% \documentclass[journal]{../sty/IEEEtran}





% Some very useful LaTeX packages include:
% (uncomment the ones you want to load)


% *** MISC UTILITY PACKAGES ***
%
%\usepackage{ifpdf}
% Heiko Oberdiek's ifpdf.sty is very useful if you need conditional
% compilation based on whether the output is pdf or dvi.
% usage:
% \ifpdf
%   % pdf code
% \else
%   % dvi code
% \fi
% The latest version of ifpdf.sty can be obtained from:
% http://www.ctan.org/pkg/ifpdf
% Also, note that IEEEtran.cls V1.7 and later provides a builtin
% \ifCLASSINFOpdf conditional that works the same way.
% When switching from latex to pdflatex and vice-versa, the compiler may
% have to be run twice to clear warning/error messages.






% *** CITATION PACKAGES ***
%
%\usepackage{cite}
% cite.sty was written by Donald Arseneau
% V1.6 and later of IEEEtran pre-defines the format of the cite.sty package
% \cite{} output to follow that of the IEEE. Loading the cite package will
% result in citation numbers being automatically sorted and properly
% "compressed/ranged". e.g., [1], [9], [2], [7], [5], [6] without using
% cite.sty will become [1], [2], [5]--[7], [9] using cite.sty. cite.sty's
% \cite will automatically add leading space, if needed. Use cite.sty's
% noadjust option (cite.sty V3.8 and later) if you want to turn this off
% such as if a citation ever needs to be enclosed in parenthesis.
% cite.sty is already installed on most LaTeX systems. Be sure and use
% version 5.0 (2009-03-20) and later if using hyperref.sty.
% The latest version can be obtained at:
% http://www.ctan.org/pkg/cite
% The documentation is contained in the cite.sty file itself.






% *** GRAPHICS RELATED PACKAGES ***
%
\ifCLASSINFOpdf
  % \usepackage[pdftex]{graphicx}
  % declare the path(s) where your graphic files are
  % \graphicspath{{../pdf/}{../jpeg/}}
  % and their extensions so you won't have to specify these with
  % every instance of \includegraphics
  % \DeclareGraphicsExtensions{.pdf,.jpeg,.png}
\else
  % or other class option (dvipsone, dvipdf, if not using dvips). graphicx
  % will default to the driver specified in the system graphics.cfg if no
  % driver is specified.
  % \usepackage[dvips]{graphicx}
  % declare the path(s) where your graphic files are
  % \graphicspath{{../eps/}}
  % and their extensions so you won't have to specify these with
  % every instance of \includegraphics
  % \DeclareGraphicsExtensions{.eps}
\fi
% graphicx was written by David Carlisle and Sebastian Rahtz. It is
% required if you want graphics, photos, etc. graphicx.sty is already
% installed on most LaTeX systems. The latest version and documentation
% can be obtained at: 
% http://www.ctan.org/pkg/graphicx
% Another good source of documentation is "Using Imported Graphics in
% LaTeX2e" by Keith Reckdahl which can be found at:
% http://www.ctan.org/pkg/epslatex
%
% latex, and pdflatex in dvi mode, support graphics in encapsulated
% postscript (.eps) format. pdflatex in pdf mode supports graphics
% in .pdf, .jpeg, .png and .mps (metapost) formats. Users should ensure
% that all non-photo figures use a vector format (.eps, .pdf, .mps) and
% not a bitmapped formats (.jpeg, .png). The IEEE frowns on bitmapped formats
% which can result in "jaggedy"/blurry rendering of lines and letters as
% well as large increases in file sizes.
%
% You can find documentation about the pdfTeX application at:
% http://www.tug.org/applications/pdftex





% *** MATH PACKAGES ***
%
%\usepackage{amsmath}
% A popular package from the American Mathematical Society that provides
% many useful and powerful commands for dealing with mathematics.
%
% Note that the amsmath package sets \interdisplaylinepenalty to 10000
% thus preventing page breaks from occurring within multiline equations. Use:
%\interdisplaylinepenalty=2500
% after loading amsmath to restore such page breaks as IEEEtran.cls normally
% does. amsmath.sty is already installed on most LaTeX systems. The latest
% version and documentation can be obtained at:
% http://www.ctan.org/pkg/amsmath





% *** SPECIALIZED LIST PACKAGES ***
%
%\usepackage{algorithmic}
% algorithmic.sty was written by Peter Williams and Rogerio Brito.
% This package provides an algorithmic environment fo describing algorithms.
% You can use the algorithmic environment in-text or within a figure
% environment to provide for a floating algorithm. Do NOT use the algorithm
% floating environment provided by algorithm.sty (by the same authors) or
% algorithm2e.sty (by Christophe Fiorio) as the IEEE does not use dedicated
% algorithm float types and packages that provide these will not provide
% correct IEEE style captions. The latest version and documentation of
% algorithmic.sty can be obtained at:
% http://www.ctan.org/pkg/algorithms
% Also of interest may be the (relatively newer and more customizable)
% algorithmicx.sty package by Szasz Janos:
% http://www.ctan.org/pkg/algorithmicx




% *** ALIGNMENT PACKAGES ***
%
%\usepackage{array}
% Frank Mittelbach's and David Carlisle's array.sty patches and improves
% the standard LaTeX2e array and tabular environments to provide better
% appearance and additional user controls. As the default LaTeX2e table
% generation code is lacking to the point of almost being broken with
% respect to the quality of the end results, all users are strongly
% advised to use an enhanced (at the very least that provided by array.sty)
% set of table tools. array.sty is already installed on most systems. The
% latest version and documentation can be obtained at:
% http://www.ctan.org/pkg/array


% IEEEtran contains the IEEEeqnarray family of commands that can be used to
% generate multiline equations as well as matrices, tables, etc., of high
% quality.




% *** SUBFIGURE PACKAGES ***
%\ifCLASSOPTIONcompsoc
%  \usepackage[caption=false,font=normalsize,labelfont=sf,textfont=sf]{subfig}
%\else
%  \usepackage[caption=false,font=footnotesize]{subfig}
%\fi
% subfig.sty, written by Steven Douglas Cochran, is the modern replacement
% for subfigure.sty, the latter of which is no longer maintained and is
% incompatible with some LaTeX packages including fixltx2e. However,
% subfig.sty requires and automatically loads Axel Sommerfeldt's caption.sty
% which will override IEEEtran.cls' handling of captions and this will result
% in non-IEEE style figure/table captions. To prevent this problem, be sure
% and invoke subfig.sty's "caption=false" package option (available since
% subfig.sty version 1.3, 2005/06/28) as this is will preserve IEEEtran.cls
% handling of captions.
% Note that the Computer Society format requires a larger sans serif font
% than the serif footnote size font used in traditional IEEE formatting
% and thus the need to invoke different subfig.sty package options depending
% on whether compsoc mode has been enabled.
%
% The latest version and documentation of subfig.sty can be obtained at:
% http://www.ctan.org/pkg/subfig




% *** FLOAT PACKAGES ***
%
%\usepackage{fixltx2e}
% fixltx2e, the successor to the earlier fix2col.sty, was written by
% Frank Mittelbach and David Carlisle. This package corrects a few problems
% in the LaTeX2e kernel, the most notable of which is that in current
% LaTeX2e releases, the ordering of single and double column floats is not
% guaranteed to be preserved. Thus, an unpatched LaTeX2e can allow a
% single column figure to be placed prior to an earlier double column
% figure.
% Be aware that LaTeX2e kernels dated 2015 and later have fixltx2e.sty's
% corrections already built into the system in which case a warning will
% be issued if an attempt is made to load fixltx2e.sty as it is no longer
% needed.
% The latest version and documentation can be found at:
% http://www.ctan.org/pkg/fixltx2e


%\usepackage{stfloats}
% stfloats.sty was written by Sigitas Tolusis. This package gives LaTeX2e
% the ability to do double column floats at the bottom of the page as well
% as the top. (e.g., "\begin{figure*}[!b]" is not normally possible in
% LaTeX2e). It also provides a command:
%\fnbelowfloat
% to enable the placement of footnotes below bottom floats (the standard
% LaTeX2e kernel puts them above bottom floats). This is an invasive package
% which rewrites many portions of the LaTeX2e float routines. It may not work
% with other packages that modify the LaTeX2e float routines. The latest
% version and documentation can be obtained at:
% http://www.ctan.org/pkg/stfloats
% Do not use the stfloats baselinefloat ability as the IEEE does not allow
% \baselineskip to stretch. Authors submitting work to the IEEE should note
% that the IEEE rarely uses double column equations and that authors should try
% to avoid such use. Do not be tempted to use the cuted.sty or midfloat.sty
% packages (also by Sigitas Tolusis) as the IEEE does not format its papers in
% such ways.
% Do not attempt to use stfloats with fixltx2e as they are incompatible.
% Instead, use Morten Hogholm'a dblfloatfix which combines the features
% of both fixltx2e and stfloats:
%
% \usepackage{dblfloatfix}
% The latest version can be found at:
% http://www.ctan.org/pkg/dblfloatfix




%\ifCLASSOPTIONcaptionsoff
%  \usepackage[nomarkers]{endfloat}
% \let\MYoriglatexcaption\caption
% \renewcommand{\caption}[2][\relax]{\MYoriglatexcaption[#2]{#2}}
%\fi
% endfloat.sty was written by James Darrell McCauley, Jeff Goldberg and 
% Axel Sommerfeldt. This package may be useful when used in conjunction with 
% IEEEtran.cls'  captionsoff option. Some IEEE journals/societies require that
% submissions have lists of figures/tables at the end of the paper and that
% figures/tables without any captions are placed on a page by themselves at
% the end of the document. If needed, the draftcls IEEEtran class option or
% \CLASSINPUTbaselinestretch interface can be used to increase the line
% spacing as well. Be sure and use the nomarkers option of endfloat to
% prevent endfloat from "marking" where the figures would have been placed
% in the text. The two hack lines of code above are a slight modification of
% that suggested by in the endfloat docs (section 8.4.1) to ensure that
% the full captions always appear in the list of figures/tables - even if
% the user used the short optional argument of \caption[]{}.
% IEEE papers do not typically make use of \caption[]'s optional argument,
% so this should not be an issue. A similar trick can be used to disable
% captions of packages such as subfig.sty that lack options to turn off
% the subcaptions:
% For subfig.sty:
% \let\MYorigsubfloat\subfloat
% \renewcommand{\subfloat}[2][\relax]{\MYorigsubfloat[]{#2}}
% However, the above trick will not work if both optional arguments of
% the \subfloat command are used. Furthermore, there needs to be a
% description of each subfigure *somewhere* and endfloat does not add
% subfigure captions to its list of figures. Thus, the best approach is to
% avoid the use of subfigure captions (many IEEE journals avoid them anyway)
% and instead reference/explain all the subfigures within the main caption.
% The latest version of endfloat.sty and its documentation can obtained at:
% http://www.ctan.org/pkg/endfloat
%
% The IEEEtran \ifCLASSOPTIONcaptionsoff conditional can also be used
% later in the document, say, to conditionally put the References on a 
% page by themselves.




% *** PDF, URL AND HYPERLINK PACKAGES ***
%
%\usepackage{url}
% url.sty was written by Donald Arseneau. It provides better support for
% handling and breaking URLs. url.sty is already installed on most LaTeX
% systems. The latest version and documentation can be obtained at:
% http://www.ctan.org/pkg/url
% Basically, \url{my_url_here}.




% *** Do not adjust lengths that control margins, column widths, etc. ***
% *** Do not use packages that alter fonts (such as pslatex).         ***
% There should be no need to do such things with IEEEtran.cls V1.6 and later.
% (Unless specifically asked to do so by the journal or conference you plan
% to submit to, of course. )


% correct bad hyphenation here
\hyphenation{op-tical net-works semi-conduc-tor}
\usepackage[spanish]{babel}
\usepackage{url}
\usepackage{graphicx}
\graphicspath{ {images/} }
\usepackage[utf8]{inputenc}
\usepackage{multirow}


\begin{document}
%
% paper title
% Titles are generally capitalized except for words such as a, an, and, as,
% at, but, by, for, in, nor, of, on, or, the, to and up, which are usually
% not capitalized unless they are the first or last word of the title.
% Linebreaks \\ can be used within to get better formatting as desired.
% Do not put math or special symbols in the title.
\title{Procesamiento paralelo de retrasos en los recorridos de las lineas del sistema de trasporte Metropolitano para la ciudad de Montevideo}
%
%
% author names and IEEE memberships
% note positions of commas and nonbreaking spaces ( ~ ) LaTeX will not break
% a structure at a ~ so this keeps an author's name from being broken across
% two lines.
% use \thanks{} to gain access to the first footnote area
% a separate \thanks must be used for each paragraph as LaTeX2e's \thanks
% was not built to handle multiple paragraphs
%

\author{Gabriel Acosta y Emiliano Gonzalez \\
Facultad de Ingeniería, Universidad de la República\\

E-mail \{gabriel.acosta.soria, gonzalez.emiliano\}@fing.edu.uy \\        }
% \thanks{M. Shell was with the Department
% of Electrical and Computer Engineering, Georgia Institute of Technology, Atlanta,
% GA, 30332 USA e-mail: (see http://www.michaelshell.org/contact.html).}% <-this % stops a space
% \thanks{J. Doe and J. Doe are with Anonymous University.}% <-this % stops a space
% \thanks{Manuscript received April 19, 2005; revised August 26, 2015.}}

% note the % following the last \IEEEmembership and also \thanks - 
% these prevent an unwanted space from occurring between the last author name
% and the end of the author line. i.e., if you had this:
% 
% \author{....lastname \thanks{...} \thanks{...} }
%                     ^------------^------------^----Do not want these spaces!
%
% a space would be appended to the last name and could cause every name on that
% line to be shifted left slightly. This is one of those "LaTeX things". For
% instance, "\textbf{A} \textbf{B}" will typeset as "A B" not "AB". To get
% "AB" then you have to do: "\textbf{A}\textbf{B}"
% \thanks is no different in this regard, so shield the last } of each \thanks
% that ends a line with a % and do not let a space in before the next \thanks.
% Spaces after \IEEEmembership other than the last one are OK (and needed) as
% you are supposed to have spaces between the names. For what it is worth,
% this is a minor point as most people would not even notice if the said evil
% space somehow managed to creep in.



% The paper headers
\markboth{Computación de Alta Performance, Facultad de Ingeniería, Junio~2024}%
{Shell \MakeLowercase{\textit{et al.}}: Bare Demo of IEEEtran.cls for IEEE Journals}
% The only time the second header will appear is for the odd numbered pages
% after the title page when using the twoside option.
% 
% *** Note that you probably will NOT want to include the author's ***
% *** name in the headers of peer review papers.                   ***
% You can use \ifCLASSOPTIONpeerreview for conditional compilation here if
% you desire.




% If you want to put a publisher's ID mark on the page you can do it like
% this:
%\IEEEpubid{0000--0000/00\$00.00~\copyright~2015 IEEE}
% Remember, if you use this you must call \IEEEpubidadjcol in the second
% column for its text to clear the IEEEpubid mark.



% use for special paper notices
%\IEEEspecialpapernotice{(Invited Paper)}




% make the title area
\maketitle

% As a general rule, do not put math, special symbols or citations
% in the resumen or keywords.
\begin{abstract}
El presente artículo trata sobre el análisis de los retrasos en los horarios de las líneas de autobuses, del sistema de transporte en la ciudad de Montevideo, durante el mes de junio del año 2024. Se propone entregar una serie de indicadores sobre la demora en los servicios y la cantidad de personas que afecta este atraso. Como es un gran volumen de datos a procesar, se utilizan técnicas de programación paralela y se utilizarán para ello, las maquinas existentes en las salas de computación de la Facultad de Ingeniería.
\end{abstract}



% For peer review papers, you can put extra information on the cover
% page as needed:
% \ifCLASSOPTIONpeerreview
% \begin{center} \bfseries EDICS Category: 3-BBND \end{center}
% \fi
%
% For peerreview papers, this IEEEtran command inserts a page break and
% creates the second title. It will be ignored for other modes.
\IEEEpeerreviewmaketitle



\section{Descripción del problema}
% The very first letter is a 2 line initial drop letter followed
% by the rest of the first word in caps.
% 
% form to use if the first word consists of a single letter:
% \IEEEPARstart{A}{demo} file is ....
% 
% form to use if you need the single drop letter followed by
% normal text (unknown if ever used by the IEEE):
% \IEEEPARstart{A}{}demo file is ....
% 
% Some journals put the first two words in caps:
% \IEEEPARstart{T}{his demo} file is ....
% 
% Here we have the typical use of a "T" for an initial drop letter
% and "HIS" in caps to complete the first word.
\IEEEPARstart{A}{} partir de los principios para el manejo de los datos abiertos en el gobierno, \cite{principios} \cite{datos-abiertos} la intendencia de Montevideo (IM) pone a disposición de la ciudadanía la mayor parte de los datos públicos que ésta maneja, para que puedan ser procesados por cualquier ciudadano, investigador, periodista, universidades y organizaciones nacionales y extranjeras \cite{areas-tematicas}. Estos datos figuran tanto en la página de la IM \cite{ckan}, como en el catálogo de datos abiertos del Gobierno uruguayo \cite{catalogo-datos}.

% You must have at least 2 lines in the paragraph with the drop letter
% (should never be an issue)
A partir de estos mismos datos, la IM publica datos estadísticos sobre el desarrollo de ciertas áreas de interés \cite{montevidata}, donde la movilidad en todos su aspectos ha sido relevante el los últimos años \cite{movilidad}. En particular, con la inclusión del sistema de transporte público metropolitano STM y su consiguiente digitalización y utilización de tarjetas de tipo MIFARE \cite{mifare}, sumado a la utilización de GPS para el posicionamiento de los vehículos, han ofrecido muchas posibilidades de  procesamiento  posterior a partir de las ventas y el seguimiento de los autobuses en su recorrido por la ciudad.

A partir de los datos abiertos publicados y de la captura de datos mediante la API propuesta por la IM \cite{api-im}, se propone calcular el retraso de los buses del sistema STM en Montevideo durante el mes de junio de 2024, luego relacionar estos datos con los datos de la ventas de boletos en junio del mismo año, con el fin de lograr inferir a qué cantidad de personas afecta ese retraso. Otras posibles métricas a mostrar podrían ser las paradas con más atraso, la distribución de los atrasos en el correr del día o en el mes, entre otras.

Realizamos a partir de la API de la IM, una captura cada 20 segundos de la ubicación de los buses en tiempo real en todo el mes de junio de 2024. A su vez, capturamos diariamente los datos de los horarios estimados de pasada, por cada parada para cada variante y por tipo de día (hábil, sábado y domingo). Esto lo hacemos por cualquier cambio eventual en los horarios al correr del mes.

Además, vamos a utilizar los datos de ventas de boletos en el mes de junio del mismo año. Con estos datos y el procesamiento de retrasos de buses, podemos saber cuántas personas por parada por día son afectadas por este retraso.

% \subsection{Subsection Heading Here}
% Subsection text here.

% needed in second column of first page if using \IEEEpubid
%\IEEEpubidadjcol

% \subsubsection{Subsubsection Heading Here}
% Subsubsection text here.


% An example of a floating figure using the graphicx package.
% Note that \label must occur AFTER (or within) \caption.
% For figures, \caption should occur after the \includegraphics.
% Note that IEEEtran v1.7 and later has special internal code that
% is designed to preserve the operation of \label within \caption
% even when the captionsoff option is in effect. However, because
% of issues like this, it may be the safest practice to put all your
% \label just after \caption rather than within \caption{}.
%
% Reminder: the "draftcls" or "draftclsnofoot", not "draft", class
% option should be used if it is desired that the figures are to be
% displayed while in draft mode.
%
%\begin{figure}[!t]
%\centering
%\includegraphics[width=2.5in]{myfigure}
% where an .eps filename suffix will be assumed under latex, 
% and a .pdf suffix will be assumed for pdflatex; or what has been declared
% via \DeclareGraphicsExtensions.
%\caption{Simulation results for the network.}
%\label{fig_sim}
%\end{figure}

% Note that the IEEE typically puts floats only at the top, even when this
% results in a large percentage of a column being occupied by floats.


% An example of a double column floating figure using two subfigures.
% (The subfig.sty package must be loaded for this to work.)
% The subfigure \label commands are set within each subfloat command,
% and the \label for the overall figure must come after \caption.
% \hfil is used as a separator to get equal spacing.
% Watch out that the combined width of all the subfigures on a 
% line do not exceed the text width or a line break will occur.
%
%\begin{figure*}[!t]
%\centering
%\subfloat[Case I]{\includegraphics[width=2.5in]{box}%
%\label{fig_first_case}}
%\hfil
%\subfloat[Case II]{\includegraphics[width=2.5in]{box}%
%\label{fig_second_case}}
%\caption{Simulation results for the network.}
%\label{fig_sim}
%\end{figure*}
%
% Note that often IEEE papers with subfigures do not employ subfigure
% captions (using the optional argument to \subfloat[]), but instead will
% reference/describe all of them (a), (b), etc., within the main caption.
% Be aware that for subfig.sty to generate the (a), (b), etc., subfigure
% labels, the optional argument to \subfloat must be present. If a
% subcaption is not desired, just leave its contents blank,
% e.g., \subfloat[].


% An example of a floating table. Note that, for IEEE style tables, the
% \caption command should come BEFORE the table and, given that table
% captions serve much like titles, are usually capitalized except for words
% such as a, an, and, as, at, but, by, for, in, nor, of, on, or, the, to
% and up, which are usually not capitalized unless they are the first or
% last word of the caption. Table text will default to \footnotesize as
% the IEEE normally uses this smaller font for tables.
% The \label must come after \caption as always.
%
%\begin{table}[!t]
%% increase table row spacing, adjust to taste
%\renewcommand{\arraystretch}{1.3}
% if using array.sty, it might be a good idea to tweak the value of
% \extrarowheight as needed to properly center the text within the cells
%\caption{An Example of a Table}
%\label{table_example}
%\centering
%% Some packages, such as MDW tools, offer better commands for making tables
%% than the plain LaTeX2e tabular which is used here.
%\begin{tabular}{|c||c|}
%\hline
%One & Two\\
%\hline
%Three & Four\\
%\hline
%\end{tabular}
%\end{table}


% Note that the IEEE does not put floats in the very first column
% - or typically anywhere on the first page for that matter. Also,
% in-text middle ("here") positioning is typically not used, but it
% is allowed and encouraged for Computer Society conferences (but
% not Computer Society journals). Most IEEE journals/conferences use
% top floats exclusively. 
% Note that, LaTeX2e, unlike IEEE journals/conferences, places
% footnotes above bottom floats. This can be corrected via the
% \fnbelowfloat command of the stfloats package.



\section{Implementación de la Solución}
\subsection{Justificación de usar HPC}
Realizamos una captura cada 20 segundos durante 30 días. Esto nos da un total de 129600 capturas en total y un promedio de unos 400 buses por captura (máximos de 1500 buses mínimos de 50). Por lo tanto, se necesita procesar unos 52 millones de registros aproximadamente para el mes de junio.
En estos registros figuran la variante de la línea, la hora de salida de la terminal (llamado por la IM frecuencia), el número de coche, la ubicación geográfica entre otros datos relevantes. Además, descargamos diariamente los archivos de horarios que se encuentra en la web de datos abiertos de la IM, por lo que contamos con 30 archivos de horarios distintos (uno por cada día del mes de junio del 2024). Es de hacer notar, que cada archivo de horarios cuenta con unas 2.5 millones de líneas. Recorriendo los datos de capturas y horarios, se pretende calcular el atraso con la hora de pasada prevista para cada parada. Encontrar la hora de pasada por cada parada, para determinada variante frecuencia y día en particular, implica un esfuerzo computacional considerable, lo que lleva a un tiempo prolongado de ejecución, si este se realiza de forma secuencial. Otro detalle adicional a considerar, es que la ubicación geográfica de las capturas de GPS se encuentra en formato WGS84 (EPSG:4326), mientras que los datos de los horarios junto con las ubicaciones de las paradas se encuentra en formato UTM 21S (EPSG:32721). Esta transformación necesaria, agrega más tiempo al procesamiento.

\subsection{Pre-procesamiento de los datos}
Los archivos que la IM brinda para los horarios por paradas no están geo-referenciados. Para no aumentar el cómputo, se agregan a estos archivos, los campos de las coordenadas que brinda la IM para cada parada que figura en los datos abiertos de la IM. \cite{paradas}

\subsection{Estrategia de resolución}
Para el análisis utilizaremos estos dos términos: 

\begin{itemize}
  \item VFD (Variante, frecuencia,  día): \\Representa de forma única, una variante de bus, una hora de partida desde el origen (llamada por la IM frecuencia) y una fecha determinada. Por ejemplo: variante \#4735 (191 - Punta Carretas), 19020 (19:02 horas), 01/06/2024, corresponde al VFD 4735-19020-01/06/2024, estos datos se procesarán desde el archivo de capturas.
  \item VFT (Variante, frecuencia, tipo de día): \\Representa de forma única una variante de bus, una hora de partida desde el origen, y un tipo de día (hábil=1, sábado=2, domingo=3). Por ejemplo: 4735-19020-1, este dato se procesará desde el archivo de horarios.
\end{itemize}

\begin{figure}
  \centering
  \includegraphics[width=9cm]{diagrama}
  \caption{Esquema de resolución}
\end{figure}

En el proceso de capturas de ubicaciones de buses, generamos un archivo CSV con las capturas en el rango de una hora. Deberíamos contar con un total de 720 archivos (24 horas por 30 días). Sin embargo, debido a caídas en el sistema que creamos para la captura de datos, sumados a fallas en la API de la IM, se capturaron 702 archivos completos y 3 archivos parciales, resultando en una falta de 5 archivos de capturas lo que resulta en un 98\% de capturas aproximadamente. \\

El esquema de la figura 1 retrata la estrategia de resolución. Para la implementación se utilizó un modelo maestro/esclavo usando un esquema de descomposición de dominio. Esta implementación se realizo en C con la biblioteca MPI.

Los datos se agrupan en directorios,para las capturas es uno por cada día (al cual le llamamos bloque), por su parte los horarios se guardaran en un directorio, donde el nombre denota el día al que pertenece. Cada proceso procesara al menos un bloque, con su correspondiente horario. El proceso master divide de forma equitativa, la ruta de los directorios entre los procesos esclavos, utilizando para esto MPI Scatter.
Tanto el master como los esclavos primero agrupan las capturas de sus bloques asignados por VFD, y para cada VFD consiguen los datos del VFT , desde el archivo de horarios correspondiente, generando archivos temporales que luego serán utilizado por el algoritmo caja negra. Este algoritmo consta de un script en Python que utiliza bibliotecas de PyQGIS para el calculo de atrasos. Este algoritmo, retorna los atrasos correspondientes para cada parada en las que existen capturas cercanas y luego elimina todo tipo de capa o memoria que pudo haber utilizado el QGIS.
Al final del procesamiento de cada bloque, los esclavos retornan sus resultados al master, pero antes realizan un pre-procesamiento para sacar datos incongruentes que surgen a partir del calculo de los atrasos. 

Una vez procesados los atrasos de todo el mes, el proceso master sumariza los retrasos y los cruza con los datos de las ventas de boletos del mismo mes,para cada variante, parada y día, con el fin de calcular la cantidad de personas afectadas en cada parada por los retrasos en las lineas. Inicialmente consideramos procesar los archivos de ventas de Junio de años anteriores, debido al atraso en la publicación de los datos. Afortunadamente, pudimos contar con las ventas de pasajes en el mes de Junio del presente año. Con estos números, se tiene una idea de a que cantidad gente afecta los atrasos.

Los datos obtenidos desde la API se dividieron por bloques de 24 horas. Estos bloques son distribuidos dependiendo de la cantidad de procesos con los que se cuente, pero para poder controlar el correcto funcionamiento de los nodos, con las pruebas realizadas, se determino que, la cantidad de bloques (días) a distribuir sera de una unidad. Con las medidas realizadas, se llega a que la contabilizacion de las 24 horas de un día, insume aproximadamente media hora.

Cada proceso deberá ordenar los datos por VFD antes de procesarlos. Esto hará que cada uno de ellos cuente con datos totales o parciales para determinado VFD. Luego del ordenamiento se realizará el procesamiento de los datos.

Inicialmente, se pensó en separar por bloques desde las 3 AM (aproximadamente), los datos serian divididos entre bloques de 24 horas hasta las 2 AM del día siguiente, pero debido al poco tiempo en que realizamos las pruebas, optamos por separar en bloques (carpetas) con las 24 horas del día. De igual manera, los horarios de los buses cambian todos los días, así que generaremos un archivo para cada día. Como el archivo de Python procesara todo un día, al final del procesamiento hecho por el código en C, los VFD parciales serán procesados por el master una vez reciba los datos de los esclavos.

Se implementaron dos algoritmos: uno para procesar el
conjunto de datos en cada proceso (ordenar en VFDs y obtener VFTs), esta implementación se realiza en C y luego se utiliza para el cálculo de la demora, un script de Python utilizando librerías de QGIS \cite{qgis}, que realiza las siguientes tareas: transformar las coordenadas de las paradas del VFT de los sistemas WGS84 a UTM 21S, determinar qué paradas del VFT serán cubiertas por el recorrido del VFD (ya que pueden ser parciales). Luego, para estas paradas válidas se calcula la hora de pasada utilizando para ellos los dos registros más cercanos, retornando así la demora en minutos para cada parada e incluyendo datos relevantes para un posterior análisis. 

\subsection{Post-procesamiento de los resultados}
Una vez el nodo termina de analizar el bloque, la información la procesa y se la envía al master. Este a su vez, fusiona los datos para devolver la demora para cada parada para cada día, variante y enlaza estos datos con la cantidad de vetas por parada, para de esta forma estimar como afecta este retraso.

Es importante hacer notar que todo este mecanismo aplica cuando la aplicación se corre con mas de un proceso. En el caso de ejecutar con un único proceso, su ejecución sera secuencial.

\section{Análisis experimental}
\subsection{Escalabilidad}
\begin{itemize}
\item Comunicación entre procesos:\\
La comunicación entre procesos se realiza al estilo rendevous, donde cada esclavo envía sus resultados calculados para cada día, utilizando un barrier para sincronizar los procesos y luego realizando envíos bloqueantes al master. 
\item Balance de carga:\\
Se utiliza una arquitectura de maestro-esclavo, donde el maestro distribuye los bloques de forma equitativa entre los esclavos utilizando MPI Scatter. 
\item Granularidad:\\
La granularidad esta definida por día de capturas 
y horarios, con el fin de minimizar los cálculos parciales de atrasos en el recorrido y evitar posibles números erróneos.
\end{itemize}
\subsection{Rendimiento}
Para evaluar el rendimiento del sistema se calcularán diferentes métricas:
\begin{itemize}
\item Tiempo de ejecución en minutos (TE).
\item Speedup logarítmico.
\item Eficiencia.
\end{itemize}
Como ya se menciono anteriormente, en el caso de ejecución con un único proceso, esta se realiza de forma secuencial lo que implica que las métricas se basen en calculo de Speedup algorítmico y no de paralelidad.

\subsection{Evaluación experimental}
Cada una de estas métricas se calcularon sobre diversas cantidades de procesos. En la siguiente tabla de la figura \ref{fig:table} se observan los resultados de los datos:

\begin{figure}[h]
\begin{tabular}{ |c|c|c|c|c|c| } 
\hline
Nodos & Procesos & TE & Speedup & Eficiencia \\
\hline
1 & 1 & 293 & 1 & 1  \\
\hline
2 & 2 & 276 & 1,06 & 0,53  \\
\hline
3 & 3 & 261 & 1,12 & 0,37  \\
\hline
5 & 5 & 228 & 1,29 & 0,26  \\
\hline
6 & 6 & 214 & 1,36 & 0,23  \\
\hline
15 & 15 & 170 & 1,72 & 0,11  \\
\hline
30 & 30 & 104 & 2,8 & 0,09  \\
\hline
\end{tabular}
\caption{Tabla de resultado de los datos}
\label{fig:table}
\end{figure}

En el gráfico de la figura \ref{fig:speedup}, podemos observar los valores de speedup calculados versus la función lineal:

\begin{figure}[h]
    \centering
    \includegraphics[width=0.48\textwidth]{speedup}
    \caption{Gráfica del speedup}
    \label{fig:speedup}
\end{figure}

\subsection{Discusión de resultados experimentales}
Notamos que corremos un proceso por nodo debido a que el algoritmo de Python que corre QGIS consume una gran cantidad de recursos de memoria (memory-bound). Además, la cantidad de procesos/nodos es divisor de la cantidad de bloques asignados debido a la limitación del MPI Scatter convencional. Otro detalle a mencionar es que, dado que la granularidad esta dada a nivel de bloques, correr la aplicación con mas de 30 procesos seria contraproducente debido que los procesos quedarían ociosos ya que no tendrían tareas a realizar.

Mediante los cálculos de tiempo estimado, speedup y eficiencia notamos que el rendimiento del sistema no mejora de manera deseada a medida que aumenta la cantidad de procesos. Creemos que esto sucede en parte por estas razones:
\begin{itemize}
\item La imposibilidad de correr mas de un proceso por nodo debido al gran consumo de memoria del algoritmo en PyQGIS.
\item El overhead de comunicación entre los esclavos y el master para recolectar los retrasos.
\item Cálculos extra realizados por el proceso master que se podrían haber paralelizado como por ejemplo la asociación de retrasos con ventas del mismo mes, debido a que lamentablemente por falta de tiempo, no pudimos implementarlo de forma paralela.
\end{itemize}

\subsection{Posibles mejoras a realizar}
Dentro de las posibles mejoras a realizar para mejorar la performance del sistema, se encuentran:
\begin{itemize}
\item Disminuir la cantidad de datos con la cual se ejecuta el algoritmo PyQGIS con el fin de disminuir el consumo de memoria y poder ejecutar mas de un proceso por nodo, aprovechando las ventajas de la localidad.
\item Paralelizar en lo posible, tareas existentes que se realizan de forma secuencial.
\item Utilización de ventanas de memoria compartida entre procesos. Esto se intento realizar de forma fallida.
\item Utilizar MPI Gather para recolectar los resultados de los esclavos en vez de sincronización y comunicación bloqueantes del estilo rendevous.
\end{itemize}

\section{Resultados obtenidos y Conclusiones}
\subsection{Resultados del procesamiento}

\begin{figure}[h]
    \centering
    \includegraphics[width=0.48\textwidth]{Retraso2}
    \caption{Mapa de calor con los atrasos}
    \label{fig:map1}
\end{figure}
\begin{figure}[h]
    \centering
    \includegraphics[width=0.48\textwidth]{Pasajeros_v2}
    \caption{Mapa de calor con los pasajeros afectados por los retrasos}
    \label{fig:map2}
\end{figure}

Si observamos los mapas de calor de retrasos por parada de ómnibus (Figura \ref{fig:map1}) y venta de boletos en esas paradas (Figura \ref{fig:map2}), se puede ver que los focos de mayor retraso coinciden con mayores ventas.
\newline

Como muestra la Figura \ref{fig:map6}, los barrios mas afectados por los atrasos de los ómnibus son Centro, Cordón, La Unión, Aguada, La Blanqueada y Belvedere.
\newline
Y las avenidas con mas atrasos son 18 de Julio, 8 de Octubre, Agraciada y Carlos María Ramírez.
\newline

\begin{figure}[h]
    \centering
    \includegraphics[width=0.48\textwidth]{Focos_fuertes}
    \caption{Mapa de calor con focos de los atrasos}
    \label{fig:map6}
\end{figure}

\begin{figure}[h]
    \centering
    \includegraphics[width=0.48\textwidth]{graficoAtrasosCategorizado}
    \caption{Gráfico que muestra el porcentaje de personas afectadas por determinados minutos de retraso}
    \label{fig:atrasos}
\end{figure}

En la figura \ref{fig:atrasos} se muestra , para las personas afectadas por los retrasos, que porcentaje  de estas se distribuidas por grupos de atrasos. Observando la gráfica, podemos notar que la mayoría de retrasos son de menos de 5 minutos. Sin embargo, existe una cantidad considerable de retrasos de hasta 10 minutos.

\subsection{Conclusiones}
En este informe, hemos mostrado la efectividad del uso de técnicas de programación paralela para procesar grandes volúmenes de datos relacionados con los retrasos en el sistema de transporte metropolitano de Montevideo. La implementación de un enfoque distribuido en los sistemas de cómputo disponibles en la Facultad de Ingeniería permitió procesar los datos de manera más eficiente, reduciendo de cierta manera el tiempo de cálculo en comparación con enfoques secuenciales.

Los resultados obtenidos revelan patrones importantes en los retrasos de las líneas de buses, identificando tanto las zonas con atraso,como las corredores más afectadas. Estos hallazgos podrían ser útiles para la Intendencia de Montevideo y las empresas de transporte publico en la planificación de mejoras, ya sea en la red , en los horarios, y las ubicaciones de las paradas, para descongestionar el uso intensivo de estas y para con ello optimizar la puntualidad del servicio.

% if have a single appendix:
%\appendix[Proof of the Zonklar Equations]
% or
%\appendix  % for no appendix heading
% do not use \section anymore after \appendix, only \section*
% is possibly needed

% use appendices with more than one appendix
% then use \section to start each appendix
% you must declare a \section before using any
% \subsection or using \label (\appendices by itself
% starts a section numbered zero.)
%


% \appendices
% \section{Proof of the First Zonklar Equation}
% Appendix one text goes here.

% you can choose not to have a title for an appendix
% if you want by leaving the argument blank
% \section{}
% Appendix two text goes here.


% use section* for acknowledgment
% \section*{Acknowledgment}


% The authors would like to thank...


% Can use something like this to put references on a page
% by themselves when using endfloat and the captionsoff option.
\ifCLASSOPTIONcaptionsoff
  \newpage
\fi



% trigger a \newpage just before the given reference
% number - used to balance the columns on the last page
% adjust value as needed - may need to be readjusted if
% the document is modified later
%\IEEEtriggeratref{8}
% The "triggered" command can be changed if desired:
%\IEEEtriggercmd{\enlargethispage{-5in}}

% references section

% can use a bibliography generated by BibTeX as a .bbl file
% BibTeX documentation can be easily obtained at:
% http://mirror.ctan.org/biblio/bibtex/contrib/doc/
% The IEEEtran BibTeX style support page is at:
% http://www.michaelshell.org/tex/ieeetran/bibtex/
%\bibliographystyle{IEEEtran}
% argument is your BibTeX string definitions and bibliography database(s)
%\bibliography{IEEEabrv,../bib/paper}
%
% <OR> manually copy in the resultant .bbl file
% set second argument of \begin to the number of references
% (used to reserve space for the reference number labels box)
\begin{thebibliography}{1}

\bibitem{principios}
Principios para el manejo de datos abiertos en el gobierno | Intendencia de Montevideo. \\
URL: \url{https://montevideo.gub.uy/areas-tematicas/servicios-digitales/datos-abiertos/principios-para-el-manejo-de-datos-abiertos-en-el-gobierno}

\bibitem{datos-abiertos}
8 Principles of Open Government Data. \\
URL: \url{https://public.resource.org/8_principles.html}

\bibitem{areas-tematicas} 
Servicios digitales | Intendencia de Montevideo. \\
URL: \url{https://montevideo.gub.uy/areas-tematicas/servicios-digitales}

\bibitem{ckan} Portal de Datos Abiertos | Intendencia de Montevideo. \\ 
URL: \url{https://ckan.montevideo.gub.uy/  }

\bibitem{catalogo-datos} Catálogo de Datos Abiertos | GUB.UY \\
URL: \url{https://catalogodatos.gub.uy/}

\bibitem{montevidata} Datos Abiertos Ambientales | Intendencia de Montevideo. \\
URL: \url{https://montevidata.montevideo.gub.uy/}

\bibitem{movilidad} Movilidad | Intendencia de Montevideo. \\ URL: \url{https://montevidata.montevideo.gub.uy/movilidad}

\bibitem{mifare} MIFARE: Contactless Chips. \\
URL: \url{https://es.wikipedia.org/wiki/Mifare}

\bibitem{api-im} API de Transporte Publico | Intendencia de Montevideo. \\ 
URL: \url{https://api.montevideo.gub.uy/apidocs/publictransport}

\bibitem{qgis} QGIS | Sistema de Información Geográfica libre y de Código Abierto. \\ 
URL: \url{https://www.qgis.org/es/site/about/index.html}

\bibitem{paradas} Transporte colectivo: paradas, puntos de control y recorridos de omnibus. URL: \url{https://catalogodatos.gub.uy/dataset/transporte-colectivo-paradas-puntos-de-control-y-recorridos-de-omnibus}

\end{thebibliography}

% biography section
% 
% If you have an EPS/PDF photo (graphicx package needed) extra braces are
% needed around the contents of the optional argument to biography to prevent
% the LaTeX parser from getting confused when it sees the complicated
% \includegraphics command within an optional argument. (You could create
% your own custom macro containing the \includegraphics command to make things
% simpler here.)
%\begin{IEEEbiography}[{\includegraphics[width=1in,height=1.25in,clip,keepaspectratio]{mshell}}]{Michael Shell}
% or if you just want to reserve a space for a photo:

% \begin{IEEEbiography}{Michael Shell}
% Biography text here.
% \end{IEEEbiography}

% if you will not have a photo at all:
% \begin{IEEEbiographynophoto}{John Doe}
% Biography text here.
% \end{IEEEbiographynophoto}

% insert where needed to balance the two columns on the last page with
% biographies
%\newpage

% \begin{IEEEbiographynophoto}{Jane Doe}
% Biography text here.
% \end{IEEEbiographynophoto}

% You can push biographies down or up by placing
% a \vfill before or after them. The appropriate
% use of \vfill depends on what kind of text is
% on the last page and whether or not the columns
% are being equalized.

%\vfill

% Can be used to pull up biographies so that the bottom of the last one
% is flush with the other column.
%\enlargethispage{-5in}



% that's all folks
\end{document}


